\chapter{Struttura programmi}

\section{Server e client in comune}

\subsection{Funzioni comuni}

Il server, il client ed i tests condividono librerie utilizzate e codice per delle funzionalit\'a richieste comuni grazie ai files "common.h" e "common.c". 
\\
Enunciamo sinteticamente le funzioni condivise. Abbiamo una funzione per fare il parsing dell'IP ottenuto dalla riga di comando, una per trasformare le stringhe in UPPERCASE o lowercase, le fondamentali "receiveMessage()" e "sendMessage()" che servono a scambiarsi i messaggi tra server e client, una funzione "destroyMessage()" per deallocare tali messaggi, una funzione "createBanner()" che crea delle stringhe carine da stampare per suddividere l'output e facilitarne la lettura. C'\'e la funzione "disconnecterChecker()" che si occupa di inviare un messaggio di "ping" al client/server per riconoscere e gestire la disconnessione del socket. Per concludere vi \'e la funzione "itoa()" che converte un numero nella stringa di caratteri che lo rappresenta. Tali funzioni sono identiche in server, client e tests. Vi \'e poi la definizione comune di un mutex "printmutex" utilizzato per stampare messaggi composti da chiamate differenti alla funzione "fprintf()" avendo la garanzia che nel mezzo non si interpongano stampe provenienti da altri threads.
 
 \subsection{Funzioni comuni ma con differente implementazione}

 Client e server condividono il possedere un thread dedicato alla gestione dei segnali ricevuti. L'implementazione del thread differisce tra i due.
   
\subsection{Gestione segnali}

Come anticipato, vi \'e un thread dedicato alla gestione dei segnali. Questo intercetta i segnali ricevuti da gestire, attraverso un loop, con una "sigwait()". Inizialmente lo sviluppo \'e stato fatto con la registrazione dei signals handlers, ma questo esponeva ad svariati difficolt\'a dovute al fatto che il segnale pu\'o finire ad un thread casuale e sopratutto che nel suo gestore, si \'e costretti ad utilizzare un numero estremamente ridotto di funzioni che devono avere la caratteristica di essere async-signal-safe. Con l'uso di un thread dedicato e della "sigwait()" invece sappiamo gi\'a il thread che ricever\'a il segnale e possiamo gestirlo liberamente senza preoccuparci di alcuno stato.

\section{Server}

\subsection{Thread main acceptClient()}

Il server, dopo un'iniziale fase di setup, con controllo degli argomenti ricevuti da riga di comando, avvio e configurazione del thread per la gestione dei segnali, registrazione del gestore dell'unico segnale (SIGUSR1) non gestito dal predetto thread, caricamento del file dizionario e matrici (se presente), apre un socket, si mette in ascolto su IP e porta ricevuti da riga di comando, avvia il primo gioco (impostando un timer di fine partita con la chiamata "alarm()") e si mette ad accettare clients indefinitamente in un "while(1)". All'arrivo di un client (connessione), lo accetta, inizializza un nuovo elemento della lista clients, aggiungendolo a tale lista (con dovute sincronizzazioni) ed infine avvia un thread ("clientHandler()") dedicato alla gestione del giocatore, passandogli come argomento il puntatore alla propria struttura dati aggiunta alla lista.

 \subsection{Thread clientHandler()}

 Il neoavviato thread si mette in un loop "while(1)" ad attendere la ricezione di un messaggio dal client attraverso la funzione "receiveMessage()", che al suo interno utilizza la chiamata "read()", per leggere le varie componenti del messaggio dal socket utente. All'arrivo di un messaggio valido tenta di acquisire il proprio mutex. Ogni client, infatti, all'interno della propria struttura dati, ha anche un suo personale mutex "handlerequest", che acquisisce durante l'elaborazione di un messaggio ricevuto, per poi rilasciarlo al termine. Tale mutex \'e posto DOPO la lettura del messaggio, ma PRIMA della sua elaborazione e permette al seguente "signalsThread()" thread di aspettare che tutte le richieste ricevute siano elaborate prima di terminare il gioco.
 
 \subsection{Thread signalsThread()}

Quando il tempo di gioco termina, si riceve un segnale SIGALRM che verr\'a intercettato da questo thread gestore segnali ed inizier\'a la fase pi\'u complessa del progetto. Sincronizzandosi opportunamente bloccher\'a tutti i threads "clientHandler()" acquisendo ciascun loro mutex "handlerequest" dopo che avranno elaborato i messaggi ricevuti. Se non dovesse riuscire ad acquisirne uno, ritenter\'a. In questo modo si \'e scelto di dare la priorit\'a al thread "clientHandler()" che sta gestendo un messaggio, questo ci garantisce la piacevole propriet\'a che tutte le richieste ricevute PRIMA dello scadere del timer siano processate come \'e giusto che sia. A seguire, il thread "signalsThread()" abilita la pausa, modificando una variabile globale "pauseon", che in seguito istruir\'a i "clientHandler()" threads di rispondere alle richieste di conseguenza (siamo in pausa). Ad ognuno di questi threads da parte del "signalsHandler()" che sta gestendo il SIGALRM verr\'a inviato il segnale SIGUSR1 per il quale nel main era stato registrato un handler (questo segnale NON \'e gestito dal "signalsThread()" thread, ma ogni "clientHandler()" thread lo ricever\'a e gestir\'a). Lo scopo di questa azione \'e interrompere eventuali "read()" sulle quali il "clientHandler()" si dovesse trovare, perch\'e adesso (dopo esser stati sbloccati dal "signalsHandler()" thread con il rilascio del proprio mutex "handlerequest") i "clientHandler()" threads dovranno riempire la coda con i messaggi contenenti il nome e punteggio del giocatore nel campo "data" con le dovute sincronizzazioni (usando il "queuemutex"). Quando la coda sar\'a stata riempita, il "signalsThread()" thread avvier\'a il thread "scorer()" che stiler\'a la classifica finale recuperando i messaggi dalla coda ed ordinandoli per punteggio discendente con una chiamata "qsort()", creando la CSV scoreboard in formato stringa "nomeutente,puntiutente". Al termine del "scorer()" thread, il "signalsThread()" che lo avr\'a aspettato con una join, ribloccher\'a tutti i "clientHandler()" threads (sempre acquisendo ciascun loro mutex "handlerequest" scorrendo la lista giocatori), li comunicher\'a che la scoreboard CSV \'e pronta e loro (i "clientHandler()" threads), dopo esser stati tutti liberati, invieranno la scoreboard ai propri (corrispettivi) utenti gestiti con un messaggio di tipo "MSG\_PUNTI\_FINALI". A questo punto, "signalsThread()" non dovr\'a far altro che avviare il thread "gamePauseAndNewGame()" (poi torner\'a a gestire altri eventuali segnali) che eseguir\'a una "sleep()" di durata della pausa. Nel frattempo, ovviamente, tutti i "clientHandler()" threads saranno liberi di operare rispondendo liberamente alle richieste dei giocatori (tenendo conto che la pausa \'e stata abilitata). Al termine della "sleep()", "gamePauseAndNewGame()" thread, bloccher\'a nuovamente tutti i "clientHandler()" threads acquisendo "handlerequest" mutex, disabiliter\'a la pausa, imposter\'a una nuova matrice di gioco (dal file o causale), aggiorner\'a il "words\_valid" array cercando tutte le parole del dizionario nella nuova matrice, imposter\'a di conseguenza il "words\_validated" di ciascun giocatore resettandogli anche il punteggio ed infine avvier\'a un nuovo timer (nuova chiamata ad "alarm()").
 
 \subsection{Threads scorer() e gamePauseAndNewGame()}
 
 Questi threads sono solo di supporto, la loro spiegazione \'e nella sottosezione precedente ("signalsThread()").

\section{Client}
L'idea fondamentale sulla quale si \'e basato lo sviluppo del client \'e stata la semplicit\'a. Ho deciso intenzionalmente di demandare tutta la complessit\'a al server. Per far s\'i che il client si limiti solamente ad inviare richieste al server e stamparne le risposte. Un primo problema affrontato \'e stato il non sapere anticipatamente la grandezza dell'input dell'utente. Non vi sono infatti limitazioni sulla lunghezza della parola da sottomettere o del nome utente da registrare. Per poter gestire input di arbitraria dimensione si leggono "BUFFER\_SIZE" caratteri dallo STDIN con una "read()", essi vengono inseriti in un "char s[BUFFER\_SIZE]" statico, all'interno di una lista (allocata dinamicamente) concatenata di stringhe. Scegliendo cos\'i un "BUFFER\_SIZE" adeguato allocheremo sempre la dimensione di memoria che pi\'u si avvicina alla grandezza dell'input dell'utente, evitando di preallocare tanta memoria inutilmente che poi comunque se allocata staticamente potrebbe terminare (in caso di input utente grandi), con una lista di stringhe si ovvia a questo problema.
\\
Un altro concetto chiave su cui mi sono concentrato \'e stato la pulizia della stampa delle risposte del server e la successiva stampa (e gestione degli input) del prompt. Nello specifico, il problema \'e stato che quando viene ricevuta e stampata una risposta dal server, l'utente potrebbe essere nel mezzo di una digitazione e aver gi\'a inserito dei caratteri nel buffer STDIN. Per vedere il problema si pu\'o guardare il codice nel file "./Studies/C/ClientInputOutputSync/brokeninputoutput.c". Ho individuato due modi per affrontare tale fastidio:
\begin{itemize}
	\item Prevenirlo. Sviluppando il client come un singolo thread, che attende che l'utente completi l'input (premendo ENTER, che tramite '\textbackslash n' interrompe la "read()"), poi esegue l'input ed infine controlla se ci sono risposte del server da stampare e ricomincia il ciclo. Oppure, sviluppare il client come multithread e sincronizzare il thread che gestisce l'input con quello che stampa le risposte del server. Entrambe le soluzioni, hanno lo stesso difetto: la "read()" deve essere obbligatoriamente bloccante, per essere sicuri di leggere l'intero buffer STDIN.
	\item Risolverlo, lasciando che il buffer STDIN possa "sporcarsi" e pulirlo quando necessario. Sembra semplice, ma non lo \'e, ho trovato solo metodi non funzionanti su internet. 
\end{itemize}
\leavevmode
Dopo aver scelto la seconda opzione ed aver svolto innumerevoli tentativi sono riuscito a risolvere il problema in modo semplice, ricorrendo a due librerie, ma senza doverne fare un uso estensivo. Ho utilizzato "termios", vedere "./Studies/C/ClientInputOutputSync/termiostests.c" e "fcntl". Le due librerie citate servono solo a rendere la "getchar()" non bloccante e permettermi di pulire l'STDIN. 
\\
Per approfondire vedere "./Studies/C/ClientInputOutputSync/inputoutputasynctests.c".

 \subsection{Thread main inputHandler()}
 Il client, dopo un'iniziale fase di setup, con controllo degli argomenti ricevuti da riga di comando, avvio e configurazione del thread per la gestione dei segnali, si collega al server con IP e porta ricevuti da riga di comando e si occupa della gestione dell'input utente e la stampa delle risposte del server (quest'ultime vengono lette da una lista messaggi). Interrompo la "read()" molto spesso e controllo se ci sono risposte del server in lista, in caso affermativo, le stampo, ripulisco lo STDIN con una "getchar()" NON BLOCCANTE, svuoto la lista e ritorno al prompt pulito, in modo che se l'utente dovesse aver digitato qualcosa di incompleto possa riscriverlo e/o modificarlo (a seguito della risposta ricevuta e visualizzata). Invece, se non ci sono risposte dal server in lista, la "read()" riprende con lo STDIN inserito dall'utente che non si accorge di nulla. L'unico svantaggio \'e che le stampe delle risposte del server non avvengono in modo asincrono, si ottiene un effetto simile per l'utente, perch\'e la "read()" viene interrotta ogni pochi millisecondi per stampare le risposte del server. Ma questo con un po' di sviluppo aggiuntivo pu\'o esser risolto.

 \subsection{Thread responsesHandler()}
 La ricezione delle risposte e l'inserimento in lista invece \'e eseguita da questo thread ad-hoc e quindi \'e totalmente asincrono. Questo thread legge le risposte del server dal socket e le inserisce in lista messaggi. Si sincronizza con il precedente mediante il mutex "listmutex". La scoreboard CSV di fine gioco ricevuta dal server, viene elaborata e stampata in un formato visualmente gradevole, ma risulta gi\'a ordinata dal server.

 \subsection{Thread disconnecterCheckerThread()}
 Questo thread richiama la funzione comune (condivisa con il server) disconnecterChecker() periodicamente in un while(1). Tale azione non poteva essere svolta dai precedenti thread (come nel server). Questo perch\'e il mainthread attende l'input dell'utente e quindi \'e necessario interrompergli la "read()" molto spesso, e ci\'o chiamerebbe questa funzione troppe volte, riempiendo il server di messaggi inutili. Il responseshandlerthread, invece, attende le risposte del server, quindi il suo tempo di attesa \'e indefinito. Quindi questa funzione viene eseguita in un thread dedicato, ed esegue il disconnecterChecker() ogni secondo.

















