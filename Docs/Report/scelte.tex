\chapter{Scelte implementative e funzionalit\'a aggiuntive}

\begin{itemize}
    \item La risposta alla richiesta API GetPriceHistory segue il formato JSON:
	\begin{lstlisting}[language=json]
{
    "priceHistory": [
        {
            "dayGMT": "2025-01-01 00:00:00 GMT",
            "high": 100,
            "low": 90,
            "open": 95,
            "close": 98
        }, ...
    ]
}
	\end{lstlisting}
	Si ha un array dove ogni elemento contiene le statistiche di un giorno del mese. L'autenticazione \'e stata volutamente resa non necessaria per questa funzione. Il sistema elabora lo storico basandosi solo sui market order bid per congruenza, contenuti nel file detabase degli ordini. Il file "storicoOrdini.json" pu\'o esser adoperato come demo.
    \item Tutte le risposte (ma anche le richieste) JSON alle richieste API seguono esattamente lo schema previsto nell'assignment, senza la presenza di campi aggiuntivi. Quindi per gli ordini, in caso di errore, viene ritornato -1 come ID in tutte le situazioni, senza specificare il motivo del fallimento, perch\'e questo avrebbe richiesto un valore aggiuntivo nella risposta.
    \item Le richieste e le risposte API hanno delle loro classi che fungono da wrapper degli oggetti del progetto generali. Questo rende tutto pi\'u pedante, rispetto all'invio diretto degli oggetti Java con JSON, ma permette maggiore controllo e flessibilit\'a nello scegliere cosa inviare / ricevere e in che formato (con quali campi). Quindi, ad esempio, nel client, la creazione di un ordine segue il flusso: stringa del comando -\textgreater{}  oggetto Java ordine -\textgreater{} oggetto Java CreateRequest -\textgreater{} oggetto Java Request -\textgreater{} stringa JSON. Nel server si segue il flusso inverso, passando comunque dagli oggetti Request per eseguire il parsing: stringa JSON della richiesta -\textgreater{} oggetto Java Request -\textgreater{} oggetto Java CreateRequest -\textgreater{} oggetto Java ordine. L'ordine sar\'a eseguito e poi il server risponder\'a con il flusso: oggetto Java ordine -\textgreater{} oggetto Java ExecutionResponse -\textgreater{} oggetto Java Response -\textgreater{} stringa JSON risposta.
    \item Sul file database degli ordini vengono salvati, oltre agli ordini limit e market, anche gli stop. Questi per\'o, non hanno un corrispondente ordine market salvato sul file, ma lo hanno in memoria in struttura dati. Dopo l'esecuzione di un ordine stop, sul file database degli ordini ci sar\'a solo questo, in memoria invece, ci sar\'a questo ed in aggiunta il corrispondente ordine market in cui lo stop si \'e trasformato.
    \item Gli ordini stop sul file database ED ANCHE NELLE NOTIFICHE vengono salvati con il prezzo dell'ordine market che li ha eseguiti, mentre nella struttura dati in memoria, mantengono il prezzo della richiesta. Questo perch\'e un ordine stop potrebbe essere eseguito ad un prezzo molto pi\'u alto o basso, a causa di un alto slippage.
    \item Gli ordini scritti sul file database hanno come quantit\'a il valore per il quale sono stati eseguiti, mentre i loro corrispondenti in memoria, nella struttura dati, hanno il valore rimanente da eseguire.
    \item Per la scrittura dei files database per ordini ed utenti, \'e stata creata una classe FileHandler, che appende ordini ed utenti in fondo ai corrispettivi files senza riscrivere completamente tutto il contenuto JSON dei files. Questo \'e molto efficiente in caso di database grandi, perch\'e riduce il tempo richiesto dalle costose operazione di I/O. Poich\'e per gli utenti deve esser possibile modificare la password, in questo caso, la riga dell'utente nel file viene sostituita con spazi per cancellarla e l'utente con la password aggiornata scritto al termine del file.
    \item In caso di logout, non arrivano notifiche di ordini eseguiti.
    \item Per gli ID degli ordini \'e stata creata una classe ad-hoc, UniqueNumber, che crea dei numeri univoci.
    \item Presenza del comando (e richiesta API) "exit()" per uscire gracefully.
    \item Codice commentato secondo lo standard Javadoc.
    \item Supporto parziale al multimercato. Le classi nel sono state strutturate per la futura possibilit\'a di creare pi\'u mercati oltre a BTC/USD, con valute di propria scelta crypto e fiat (es: ETH/EUR) ed il relativo order book. Parziale, perch\'e per evitare di allontanarsi troppo dalle specifiche richieste le API e database supportano attualmente solo il mercato deafult che \'e BTC/USD.
    \item Avanzata coordinazione e gestione dello STDIN e STDOUT della CLI del client. L'input dell'utente \'e gestito e coordinato con l'output (le risposte e notifiche ricevute dal server), quest'ultimo viene asincronamente stampato senza interrompere la digitazione dell'utente. Implementato mediante la libreria per la creazione di CLI avanzate in Java, denominata JLine.
\end{itemize}

